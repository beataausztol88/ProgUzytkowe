\documentclass[]{beamer}
\usepackage[MeX]{polski}
%\usepackage[cp1250]{inputenc}
%\usepackage{polski}
\usepackage[utf8]{inputenc}
\beamersetaveragebackground{blue!10}
\usetheme{Warsaw}
\usecolortheme[rgb={0.1,0.5,0.7}]{structure}
\usepackage{beamerthemesplit}
\usepackage{multirow}
\usepackage{multicol}
\usepackage{array}
\usepackage{graphicx}
\usepackage{enumerate}
\usepackage{amsmath} %pakiet matematyczny
\usepackage{amssymb} %pakiet dodatkowych symboli

\title{Prezentacja na temat ogrodnictwa} \TeX a}
\date{}

\begin{document}

\frame
{
\frametitle{Bloki w \TeX u}
\begin{block}
{Przykładowy blok w \TeX u}
Tu wpisujemy treśc
\end{block}
}

\frame
{
\begin{figure}[here]
\begin{center}
\includegraphics[scele=0.4]{ogrodnictwo.jpg}
\end{center}
\end{figure}
}

\frame
{
\begin{columns}
\column{0.2\textwidth}
\includegraphics[scale=0.2]{ogrodnictwo.jpg}
\column{0.4\textwidth}
\includegraphics[scale=0.2]{ogrodnictwo.jpg}
\end{columns}
}

\frame
{
\begin{table}
\begin{tabular}{c|c|c}
$$ & $notowania jabłek$ & $notowania gruszek$ \\
\hline
$Polska$ & 2 & 3\pause \\
\hline
$USA$ & 3 & 1\pause \\
\hline
$Francja$ & 1 & 1
\end{tabular}
\end{table}
}

\frame
{
\frametitle{Produkty na targach ogrodniczych}
\begin{itemize}
\item Nasiona warzyw
\item Rośliny dojniczkowe
\item Maszyny rolnicze
\item Sadzonki
\end{itemize}
}


\end{document}
