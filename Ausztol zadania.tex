\documentclass{article}
\usepackage[a4paper,left=3.5cm,right=2.5cm,top=2.5cm,bottom=2.5cm]{geometry}
%%\usepackage[MeX]{polski}
%%\usepackage[cp1250]{inputenc}
\usepackage{polski}
\usepackage[utf8]{inputenc}
\usepackage[pdftex]{hyperref}
\usepackage{makeidx}
\usepackage[tableposition=top]{caption}
\usepackage{algorithmic}
\usepackage{graphicx}
\usepackage{enumerate}
\usepackage{multirow}
\usepackage{amsmath} %pakiet matematyczny
\usepackage{amssymb} %pakiet dodatkowych symboli
\begin{document}
ZESPÓŁ PIERSI
Bałkanica
\begin{itemize}
\item Bałkańska w żyłach płynie krew
\item Kobiety wino, taniec śpiew
\item Zasady proste w życiu mam
\item Nie rób drugiemu czego ty nie chcesz sam!
\item Łopa!
\end {itemize}
\begin{description}
\item[1] Muzyka, przyjaźń, radość, śmiech
\item[2] Życie łatwiejsze staje się
\item[3] Przynieście dla mnie wina dzban
\item[4] Potem ruszamy razem w tan
\begin{center}
Będzie,będzie zabawa!
Będzie się działo!
I znowu nocy będzie mało.
\end{center}
\end {description}
\\
\\

\left|U\right|=sub{\left|x-y\right|:x,y\inU 
\\
\\

F\subset\bigcup^{\infty}_{i=1}U_{i},gdzie0\leq\left|U_{i}\right|
\\
\\

H^{\gamma}_{\delta}=inf{\sum\left|U\right|:{U_{i} jest b - pokryciem F}
\\
\\

H^{8}\left(F\right)

\begin{center}
\begin{tabular}{|1|p{10cm}|}
\hline
Symbol&Znaczenie\\ \hline
F&Idz do przodu jeden krok o długości l i narysuj linie\\ \hline
f&Idz do przodu jeden krok o długości l i nie rysuj linni\\ \hline
+&Obróć się w lewo\\ \hline
-&Obróć się w prawo\\ \hline
\end{tabular}
\end{center}
\begin{figure}
\caption}{Uśmiechnięta buzia bez zmian}
\certering
\includegraphics{buzia.jpg}
\end{figure}
\end{document}
